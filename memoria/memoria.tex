\documentclass[a4paper,11pt]{article}
\usepackage[american]{babel}
\usepackage[utf8]{inputenc}
\usepackage{geometry}
\usepackage{booktabs}  
\usepackage{graphicx} 
\usepackage{listings}
\usepackage{amsmath,amsthm,amssymb}
\lstset{%
backgroundcolor=\color{cyan!10},
basicstyle=\ttfamily,
numbers=left,numberstyle=\scriptsize
}
\setlength{\parindent}{0cm}
\usepackage[wby]{callouts}

\title{Memoria Proyecto Final}
\author{Álvaro Beltrán y Yábir García}
\begin{document}

\maketitle

\begin{figure}[h]
\includegraphics[scale=0.3]{UGR}
\centering
\end{figure}

\newpage

\renewcommand*\contentsname{Índice}
\tableofcontents

\newpage

\section{Definición del problema a resolver y enfoque elegido.}

\section{Argumentos a favor de la elección de los modelos.}

En cuanto a los modelos lineales elegidos nos hemos decantado por Regresión Logística y Perceptron, debido a que son dos modelos que hemos estudiado en clase para ejemplos de clasificación binaria, como es nuestro caso. \\

Además parece interesante probar con el Perceptron puesto que también vamos a usar el perceptron multicapa (MLP). Y puesto que Regresión Logística ha dado tan buenos resultados en las prácticas de esta asignatura y suele funcionar muy bien en clasificación parece indispensable probar este modelo.







\end{document}
